\thispagestyle{fancy}
\chapter{Konklusion}
\label{chp:Konklusion}
Der har gennem projektperioden vist sig et behov for at teste. Flere tests har afsløret væsentlige fejl hvor der eksempelvist blev ittereret gennem lister i forkert rækkefølge eller beregninger blev returneret med værdier som havde forkerte værdier på 12. decimal. Anden itteration gav et design som er meget nemmere at udvidde med lignende krav. Fordelen ved at implementere et design som bygger på SOLID principperne har vist sine fordele. Vi har gennem dette projekt udviklet en bedre forståelse for SOLID og relevansen af unittesting. Integrationenstestene er desværre ikke implementeret, men der er tilgengæld lavet en række test ved hjælp af et eksekverbar projekt hvor en række scenarier er testet manuelt. Dette har givet os en god fornemmelse af at projektet fungere uden at vi dog har automatiseret processen.

%I forbindelse med opgaven har gruppen fået en bedre forståelse for vigtigheden af en godt og testbart design med en klar opdeling af klasserne. Dette gjorde det også forholdsvist smertefrit at håndtere de ændrede krav, der kom sent i forløbet, og det har givet gruppen et indblik i, hvad man kan blive udsat for i et virkeligt projekt. 
%
%Brugen af Jenkins var en motiverende faktor for at lave tests løbende, da man kan følge med i grafens udvikling, og det har haft stor betydning for udviklingsforløbet at projektet. 
%
%Ydermere har gruppen fået et væsentligt bedre erfaring til brugen af testframeworks i form af NUnit, DotCover og NSubstitute. Både i forhold til det rent tekniske, men i samme grad at skabe en god struktur og præcision i de udarbejdede tests.
